\documentclass[10pt, openany]{book}

\usepackage{fancyhdr}
\usepackage{imakeidx}

\usepackage{amsmath}
\usepackage{amsfonts}

\usepackage{geometry}
\geometry{letterpaper}

\usepackage{fancyvrb}
\usepackage{fancybox}

\usepackage{url}
\usepackage{gensymb}
%
% Rules to allow import of graphics files in EPS format
%
\usepackage{graphicx}
\DeclareGraphicsExtensions{.eps}
\DeclareGraphicsRule{.eps}{eps}{.eps}{}
%
% Front Matter
%
\title{Recipies}
\author{Brent Seidel \\ Phoenix, AZ}
\date{ \today }
%========================================================
%%% BEGIN DOCUMENT
\begin{document}
%
% Produce the front matter
%
\frontmatter
\maketitle
\begin{center}
This document is \copyright 2021 Brent Seidel.  All rights reserved.

\paragraph{}Note that this is a draft version and not the final version for publication.
\end{center}
\tableofcontents

\mainmatter
%----------------------------------------------------------
\chapter{Introduction}
This is a collection of recipes that I've made and liked.  I'm not a particularly skilled baker or cook, so these should be usable by most people.  I'm also a bit lazy, so these are mostly fairly easy recipes that will provide good results with minimal effort.

All of these recipes are vegetarian, but may not be vegan.  Feel free to adjust them to your needs.

\chapter{Breads}
\section{Sun-dried Tomato Basil Braided Bread}
I got inspired to make this after watching too many episodes of the Great British Baking Show.  After finding a suitable bread recipe (yeasted, kneaded), I modified it a bit for my use.  Interestingly, on the show, the mix the dry ingredients first before adding liquid while the recipes I found start with the liquid and then add flour.  It should work either way, but if you do the dry ingredients first, Paul Hollywood says to make sure to put the yeast on the opposite side of the bowl from the salt.

The quantities of the basil, Parmesan cheese, sun dried tomatoes, and garlic are not critical.  Feel free to adjust them to taste.

This recipe makes one loaf of bread.

\subsection{Ingredients}
\begin{itemize}
  \item 1 package (1/4 ounce) (8g) active dry yeast
  \item 3/4 cup warm water (110\degree to 115\degree) (Paul Hollywood says to use cool water)
  \item 1/4 cup (3-4g) minced fresh basil
  \item 1/4 cup (25g) grated Parmesan cheese
  \item 1/4 cup (56g) chopped sun dried tomatoes packed in olive oil (if not in oil, add a little bit of oil to the recipe).
  \item 1 tablespoon (8g) sugar
  \item 2 crushed cloves of garlic (5g)
  \item 1 teaspoon (4g) salt
  \item 2 cups (350g) bread flour
\end{itemize}
\subsection{Procedure}
\begin{itemize}
  \item In a large bowl, dissolve yeast in water.
  \item Stir in sugar, basil, Parmesan cheese, sun dried tomatoes, garlic, salt, and 2 cups flour.
  \item Stir in enough remaining flour to form a stiff dough.
  \item Turn onto a floured surface; knead until smooth and elastic, about 3-5 minutes.
  \item Place in a greased bowl, turning once to grease top.
  \item Cover and let rise in a warm place until doubled, about 1 hour (more if the water was not warm).
  \item Divide dough into 3 equal pieces and roll each piece into a strand about 18 inches long.
  \item Cover a baking pan with parchment paper, a silicon baking mat, or grease.
  \item Braid the strands together on the baking pan.
  \item Cover and let rise until doubled, about 1 hour.
  \item Bake at 375\degree for 35-40 minutes or until golden brown.
  \item Remove from pan to a wire rack to cool.
\end{itemize}

\end{document}
