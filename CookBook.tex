\documentclass[10pt, openany]{book}

\usepackage{fancyhdr}
\usepackage{imakeidx}

\usepackage{amsmath}
\usepackage{amsfonts}

\usepackage{geometry}
\geometry{letterpaper}

\usepackage{fancyvrb}
\usepackage{fancybox}

\usepackage{url}
\usepackage{gensymb}
%
% Rules to allow import of graphics files in EPS format
%
\usepackage{graphicx}
\DeclareGraphicsExtensions{.eps}
\DeclareGraphicsRule{.eps}{eps}{.eps}{}
%
% Front Matter
%
\title{Recipies}
\author{Brent Seidel \\ Phoenix, AZ}
\date{ \today }
%========================================================
%%% BEGIN DOCUMENT
\begin{document}
%
% Produce the front matter
%
\frontmatter
\maketitle
\begin{center}
This document is \copyright 2021 Brent Seidel.  All rights reserved.

\paragraph{}Note that this is a draft version and not the final version for publication.
\end{center}
\tableofcontents

\mainmatter
%----------------------------------------------------------
\chapter{Introduction}
This is a collection of recipes that I've made and liked.  I'm not a particularly skilled baker or cook, so these should be usable by most people.  I'm also a bit lazy, so these are mostly fairly easy recipes that will provide good results with minimal effort.

All of these recipes are vegetarian, but may not be vegan.  Feel free to adjust them to your needs.

\chapter{Breads}
\section{Sun-dried Tomato Basil Braided Bread}
I got inspired to make this after watching too many episodes of the Great British Baking Show (aka Great British Bake-off).  After finding a suitable bread recipe (yeasted, kneaded), I modified it a bit for my use.  Interestingly, on the show, the mix the dry ingredients first before adding liquid while the recipes I found start with the liquid and then add flour.  It should work either way, but if you do the dry ingredients first, Paul Hollywood says to make sure to put the yeast on the opposite side of the bowl from the salt.

The quantities of the basil, Parmesan cheese, sun dried tomatoes, and garlic are not critical.  Feel free to adjust them to taste.

This recipe makes one loaf of bread.

\subsection{Ingredients}
\begin{itemize}
  \item 1 package (1/4 ounce) (8g) active dry yeast
  \item 3/4 cup water (Paul Hollywood says to use cool water)
  \item 1/4 cup (3-4g) minced fresh basil
  \item 1/4 cup (25g) grated Parmesan cheese
  \item 1/4 cup (56g) chopped sun dried tomatoes packed in olive oil (if not in oil, add a little bit of oil to the recipe).
  \item 1 tablespoon (8g) sugar
  \item 2 crushed cloves of garlic (5g)
  \item 1 teaspoon (4g) salt
  \item 2 cups (350g) bread flour
\end{itemize}
\subsection{Procedure}
\begin{itemize}
  \item In a large bowl, dissolve yeast in water.
  \item Stir in sugar, basil, Parmesan cheese, sun dried tomatoes, garlic, and salt.  Be sure to add the sugar first and the salt last.  Once this is fairly well mixed, move on to the next step.
  \item Add the flour and mix it in.  While adding the flour, you'll probably want to shift from mixing with a spoon to using your hands.  The dough doesn't need to be a single lump before moving on.  The kneading process will help to merge everything together.
  \item Turn onto a floured surface; knead until smooth and elastic, about 3-5 minutes.
  \item Place in a greased bowl, turning once to grease top.
  \item Cover and let rise in a warm place until doubled, about 1 hour (more if the water was not warm).
  \item Divide dough into 3 equal pieces and roll each piece into a strand about 18 inches long.  Note that the dough was greased while rising, so it doesn't need a floured surface to be worked.
  \item Cover a baking pan with parchment paper, a silicon baking mat, or grease.
  \item Braid the strands together on the baking pan.
  \item Cover and let rise until doubled, about 1 hour.
  \item Bake at 375\degree{}F for 35-40 minutes or until golden brown.
  \item Remove from pan to a wire rack too cool.
\end{itemize}

%----------------------------------------------------------
\chapter{Pastries}
\section{Choux Pastry}
\label{pastry:choux}
This is the recipe for Choux pastry that can then be used for a number of things.  As such, this will end once you have the dough made.  Other recipes will tell you what to do with the dough and how to bake it.  If you are making a sweet pastry and have a bit of a sweet tooth, you can add some sugar along with the salt.

\subsection{Ingredients}
\begin{itemize}
  \item 1 cup (228g) water
  \item 6 tablespoons (81g) butter or margarine
  \item 1/4 teaspoon (2g) salt
  \item 1 cup (175g) flour
  \item 4 large eggs
\end{itemize}
\subsection{Procedure}
\begin{itemize}
  \item Cut the butter or margarine into small chunks and add to the water.
  \item Add the salt.
  \item Bring the water to a boil.
  \item Once the water is boiling and all the butter or margarine is melted, remove from heat.
  \item Add the flour while mixing and ensure that there are no lumps
  \item Return to heat and cook until the dough pulls away cleanly from the sides of the saucepan.
  \item Let the dough cool enough so that it won't cook the eggs when you add them (about 150\degree{}F or so).
  \item Break and whisk the eggs into a separate container.
  \item Add the egg a bit at a time while mixing the dough
  \item Stop when the dough is smooth and glossy.
\end{itemize}
The dough is now ready for other projects such as eclairs, profiteroles, cream puffs, and the like.

\section{Eclairs}
\subsection{Ingredients}
\begin{itemize}
  \item Choux pastry dough (see section \ref{pastry:choux})
  \item Stuff for filling
  \item Stuff for topping
\end{itemize}
\subsection{Procedure}
\begin{itemize}
  \item Place a sheet of parchment paper on a baking tray
  \item Pipe the Choux pastry dough into lines on the paper.
  \item Bake at about 375\degree{}F for 35-45 minutes.
  \item Don't open the oven to check for the first 30 minutes.
  \item After 30 minutes in the oven, prick a small hole in the shells to let the steam out.
  \item Remove from the oven when done and let cool.
  \item Using the hole pricked, you can pipe the filling into the eclair.
  \item Add whatever topping you wish.
\end{itemize}

\end{document}
